\documentclass[11pt, letterpaper]{article}   	% use "amsart" instead of "article" for AMSLaTeX format
\usepackage[left=1.0in,top=0.9in,right=1.0in,bottom=0.9in]{geometry}                		% See geometry.pdf to learn the layout options. There are lots.
%\geometry{letterpaper}                   		% ... or a4paper or a5paper or ... 
%\geometry{landscape}                		% Activate for for rotated page geometry
%\usepackage[parfill]{parskip}    		% Activate to begin paragraphs with an empty line rather than an indent
\usepackage{graphicx}				% Use pdf, png, jpg, or eps� with pdflatex; use eps in DVI mode
								% TeX will automatically convert eps --> pdf in pdflatex		
\usepackage{amssymb}

\usepackage[parfill]{parskip} % Remove paragraph indentation
\usepackage{array} % Required for boldface (\bf and \bfseries) tabular columns
\usepackage{ifthen} % Required for ifthenelse statements
\usepackage{url}
\pagestyle{plain}
\pagenumbering{roman}
%\pagestyle{empty} % Suppress page numbers

%this is needed to get the block text justified
\usepackage{ragged2e}
\usepackage{mathptmx}

%% For repeat
\usepackage{expl3}
\ExplSyntaxOn
\cs_new_eq:NN \Repeat \prg_replicate:nn
\ExplSyntaxOff

%% For chemical formula
\usepackage{mhchem}

%% Support to fix table width
\usepackage{array}
\newcolumntype{C}[1]{>{\centering\arraybackslash}m{#1}}
\newcolumntype{R}[1]{>{\raggedleft\arraybackslash}m{#1}}

%% For multi-page tables
\usepackage{longtable}

%\title{Xiaoguang Xu}
%\author{The Author}
%\date{}							% Activate to display a given date or no date

%%%%% 
\def\addressskip{\smallskip} % The space between the two address (or phone/email) lines
\def\sectionlineskip{\medskip} % The space above the horizontal line for each section 
\def\nameskip{\bigskip} % The space after your name at the top
\def\sectionskip{\medskip} % The space after the heading section

\begin{document}
%\maketitle
%\section{}
%\subsection{}
\vspace*{1.5em}
\hfil{\MakeUppercase{\bf{\Large{X}\large{iaoguang (}\Large{R}\large{ichard) }\Large{ X}\large{u}}}}\hfil

\nameskip
University of Nebraska-Lincoln \hfill  \url{http://meteo.unl.edu/~xxu}\\
Earth \& Atmospheric Sciences \hfill   Phone: (402) 805 $\cdot$ 7712\\
Bessey Hall 130  \hfill Fax: (402) 472 $\cdot$ 4917\\
Lincoln, NE 68588-0340 \hfill xxu@huskers.unl.edu\\

\vskip-1em
\hfil\Repeat{12}{$\star\quad$}$\star$\hfil

% Enlarge the space between table rows 
 \renewcommand{\arraystretch}{1.3}
\sectionskip
{\bf\large Education}
\sectionlineskip
\hrule
\begin{tabular}{p{0.3in} p{0.42in} p{5.3in}}
Ph.D. & 2015 & University of Nebraska-Lincoln, Earth and Atmospheric Sciences\\
      &      & {\small Dissertation: \textit{Retrieval of the Aerosol
                Microphysical Properties from AERONET Photopolarimetric 
                Measurements}}\\
M.S. & 2008 & Lanzhou University (China), Meteorology \\
     &       & {\small Thesis: \textit{Study of Predictability of the 
               T63L16 Climate Model}}  \\
B.S. & 2005 & Lanzhou University (China), Atmospheric Sciences\\
\end{tabular}

%% Research Experiences
\sectionskip
{\bf\large Professional Experiences}
\sectionlineskip
\hrule
\begin{tabular}{p{1.1in} p{5.3in}}
 \justifying
01/2016--present & Research Assistant Professor, University of
Nebraska-Lincoln \\
09/2015--12/2015 & Postdoctoral Scholar, University of Nebraska-Lincoln \\
09/2012--08/2015 & Graduate Student Fellow, University of Nebraska-Lincoln\\
08/2008--08/2012 & Graduate Research Assistant, University of Nebraska-Lincoln
\\
2010 Summer & Visiting Scholar, University of Colorado, Boulder \\ 
2009 Summer & Visiting Researcher, National Center for Atmospheric
Research (NCAR) \\
07/2006--07/2008  & Graduate Researcher, National Climate Center, China
Meteorological Administration \\
\end{tabular}

%%% Teaching Experience
\sectionskip
{\bf\large Teaching at Nebraska}
\sectionlineskip
\hrule
\begin{tabular}{p{0.9in} p{5.3in}}
 \justifying
2015 Fall   & Guest Lecturer, \textit{Advanced Satellite Remote
Sensing} \\
2015 Fall   & Guest Lecturer, \textit{Physical Meteorology} \\
2013 Fall   & Teaching Assistant, \textit{Environment, Energy, and
Climate} \\
2012 Fall   & Guest Lecturer, \textit{Advanced Satellite Remote
Sensing} \\
2012 Spring & Teaching Assistant, \textit{Atmospheric
Thermodynamics}\\
\end{tabular}

\sectionskip
{\bf\large Awards}
\sectionlineskip
\hrule
\begin{tabular}{p{0.9in} p{5.3in}}
 \justifying 
 2014 & \textbf{Outstanding Graduate Award}, UNL Earth \& Atmospheric Sciences \\
 2012--2015 & \textbf{NASA Earth and Space Science Fellowship} \\
 2007 & \textbf{Outstanding Conference Paper}, China Meteorological
Society 2007 Annual Meeting \\
 2004 & \textbf{Outstanding Undergraduate (3rd place)}, Lanzhou
University \\
\end{tabular} 

%% Research Grants
\sectionskip
{\bf\large Research Grants}
\sectionlineskip
\hrule
\textbf{Sensitivity analysis and recovery of dust emissions from
spectral climate signals} \quad (Xu is a Co-I, Dr. Jun Wang is the PI). \\
\hspace{.5cm} -- NASA ROSES NNH14ZDA001N-ACSCS \\
\hspace{.5cm} -- 11/01/2014 -- 10/31/2017

\textbf{Constraining Global Sources of Atmospheric Mineral Dust with
Multi-Sensor Satellite Observations and the GEOS-Chem Adjoint Model}
\quad (Xu is the recipent of the Fellowship).\\
\hspace{.5cm} -- NASA Earth and Space Science Fellowship (NESSF12) \\
\hspace{.5cm} -- 09/01/2012 -- 08/31/2015


%%% Travel Grants
\sectionskip
{\bf\large Travel Grants}
\sectionlineskip
\hrule
\begin{longtable}{p{0.4in} p{5.8in}}
 \justifying
 2015 & The JCSDA Summer Colloquium on Satellite Data Assimilation,
Fort Collins, CO, 27 July -- 7 August 2015. \\
 2014 & The MISR Science Team Meeting, Pasadena, CA, 11--12 December 2014. \\
 2013 & The Electromagnetic and Light Scattering - XIV, Lille, France,
17--21 June 2013. \\
 2013 & The 6th International GEOS-Chem Meeting, Cambridge,
MA, 6--9 May 2013. \\
 2011 & The 5th International GEOS-Chem Meeting, Cambridge,
MA, 2--5 May 2011. \\
 2009 & The Gordon Research Conference on Radiation \& Climate,
New London, NH, 6--10 July 2009.\\
 2009 & The 4th International GEOS-Chem Meeting, Cambridge,
MA, 7--10 April 2009. \\
\end{longtable}
\vskip-1em


\sectionskip
{\bf\large Journal Publications}
\sectionlineskip
\hrule
\begin{longtable}{p{0.4in} p{5.8in}}
 \justifying 
 2016 & Wang J., C. Aegerter, \textbf{X. Xu}, and J. Szykman, \textbf{Potential
application of VIIRS Day/Night Band for monitoring nighttime surface
PM2.5 air quality from space}, \textit{Atmospheric Environment}, 2016,
124, 55--63.\\
 2016 & Hou W., J. Wang, \textbf{X. Xu}, and J. Reid, \textbf{An algorithm 
for hyperspectral remote sensing of aerosols: 1. Development of 
theoretical framework}, \textit{Journal of Quantitative
Spectroscopy and Radiative Transfer}, 2016, 178, 400--415. \\
 2016 & Ding S., J. Wang, and \textbf{X. Xu}, \textbf{Polarimetric remote
sensing in O2 A and B bands: Sensitivity study and information content
analysis for vertical profile of aerosols}, \textit{Atmos. Meas. Tech.}, 2016, 9, 2077--2092. \\
 2015 & \textbf{Xu X.} and and J. Wang, \textbf{Retrieval of aerosol
microphysical properties from AERONET photopolarimetric measurements:
1. Information content analysis}, \textit{J. Geophys. Res. Atmos.}, 2015,
120, 7059--7078. \\
 2015 & \textbf{Xu X.},  J. Wang, J. Zeng, R. Spurr, X. Liu, O. Dubovik,
L. Li, Z. Li, M. I. Mishchenko, A. Siniuk, and B. N. Holben, \textbf{
Retrieval of aerosol microphysical properties from AERONET photopolarimetric
measurements: 2. A new research algorithm and case demonstration},
\textit{J. Geophys. Res. Atmos.}, 2015, 120, 7079--7098.\\ 
2014 & Wang J., \textbf{X. Xu}, S. Ding, J. Zeng, R. Spurr, X. Liu, K. Chance,
and M. Mishchenko, \textbf{A numerical testbed for remote sensing of aerosols,
and its demonstration for evaluating retrieval synergy from a geostationary 
satellite constellation of GEO-CAPE and GOES-R}, \textit{Journal of 
Quantitative Spectroscopy and Radiative Transfer}, 2014, 146(0), 510--528. \\
2013 & Meland B. S., \textbf{X. Xu}, D. Henze, and J. Wang, \textbf{Assessing
remote polarimetric measurement sensitivities to aerosol emissions using
the GEOS-Chem adjoint model}, \textit{Atmos. Meas. Tech.}, 2013, 6,
3441--3457.
\\ 
2013 & \textbf{Xu X.}, J. Wang, D. K. Henze, W. Qu, and M. Kopacz,
\textbf{Constraints on aerosol sources using GEOS-Chem adjoint and MODIS
radiances, and evaluation with multisensor (OMI, MISR) data}, \textit{J.
Geophys. Res. Atmos.}, 2013, 118(12), 6396--6413. \\
2013 & Wang C., J. Li, and \textbf{X. Xu}, \textbf{Wavelet analysis of
quasi-3-year temperature oscillations in China in last 50 years, and
predicted changes in the next 20 years}, \textit{Sciences in Cold and
Arid Regions}, 2013, 5(6), 0755--0766. \\ 
2012 & Wang J., \textbf{X. Xu}, D. K. Henze, J. Zeng, Q. Ji, S.-C.
Tsay, and J. Huang, \textbf{Top-down estimate of dust emissions through
integration of MODIS and MISR aerosol retrievals with the GEOS-Chem
adjoint model}, \textit{Geophys. Res. Lett.}, 2012, 39(8), L08802. \\
2012 & Wang C., J. Li, X. Li, and \textbf{X. Xu}, \textbf{Analysis on
quasi-periodic characteristics of precipitation in recent 50 years and
trend in next 20 years in China}, \textit{Arid Zone Research}, 2012, 1, 002. \\
2012 & Wang C., J. Li, and \textbf{X. Xu}, \textbf{Universality of quasi-3-year
period of temperature in Last 50 years and change in next 20 years in
China}, \textit{Plateau Meteorology}, 2012, 31(1), 126--136. \\
2010 & Wang J., \textbf{X. Xu}, R. Spurr, Y. Wang, and E. Drury,
\textbf{Improved algorithm for MODIS satellite retrievals of aerosol optical
thickness over land in dusty atmosphere: Implications for air quality
monitoring in China}, \textit{Remote Sensing of Environment}, 2010, 114(11),
2575--2583. \\
2009 & \textbf{Xu X.}, W. Li, H. Ren, and P. Zhang, Distribution of
prediction capacity of T63L16 model for medium-range forecast at
different spatial scales, \textit{Acta Meteorologica Sinica}, 2009, 67(6),
992--1001.\\
\end{longtable} 
\vskip-1em
\quad Total citation: \textbf{182} (Google Scholar Citations as of June 2016) 


%%% Recent Conference Presentations (Oral) 
\sectionskip
{\bf\large Conference Talks}
\sectionlineskip
\hrule
\begin{longtable}{p{0.4in} p{5.8in}}
 \justifying
 2016 & Wang J., X. Xu, D. Henze, and L. Zhang, Assessing Information
Content of CLARREO Measurements to Size-Dependent Dust Emissions: An
OSSE Study, \textit{CLARREO Science Definition Team Meeting}, Ann Arbor,
MI, 10--12 May 2016. \\
 2015 & Wang J., S. Ding, and \textbf{X. Xu}, Polarimetric remote
sensing in O2 A and B bands: Sensitivity study and information content
analysis for vertical profile of aerosols, A21L-06, \textit{AGU 2015 Fall
Meeting}, San Francisco, CA, 14--18 December 2015. \\
 2015 & Wang J., Y. Yue, \textbf{X. Xu}, Y. Liu, R. Levy, J. J. Szykman,
and R. Holz, Will ensemble approach improve surface PM2.5 estimate from
space? \textit{MODIS/VIIRS Science Team Meeting}, Silver Spring, MD,
18--22 May 2015. \\
 2015 & Wang J., \textbf{X. Xu}, Y. Wang, D. Henze, and L. Zhang, 
 Dust Emission Optimization with Satellite Remote Sensing: Application
to CLAEERO, \textit{CLARREO Science Definition Team Meeting}, Lawrence
Berkeley National Laboratory, CA, 28--30 April 2015. \\
 2014 & \textbf{Xu X.}, J. Wang, J. Zeng, R. Spurr, X. Liu, O. Dubovik, Z. Li,
L. Li, B. Holben, and M. Mishchenko, An algorithm for retrieving fine
and coarse aerosol microphysical properties from AERONET-type
photopolarimetric measurements, A54B-04, \textit{AGU 2014 Fall Meeting}, San
Francisco, CA, 15--19 December 2014. \\
 2014 & \textbf{Xu X.}, J. Wang, Y. Wang, and D. Henze, Adjoint Inversion of 
Atmospheric Dust Sources with the MODIS and MISR Observations, 
\textit{2014 MISR Science Team Meeting}, Pasadena, CA, 11--12 December 2014. \\
 2014 & Henze D., L. Zhang, L. Zhu, \textbf{X. Xu}, J. Wang, K. Cady-Pereira,
M. Shephard, J. Bash, C. Lee, and R. Martin, Keynote: Remote Sensing
Constraints on Aerosol Sources and Impacts, \textit{Goldschmidt2014}, Sacramento,
CA, 8--13 June 2014. \\
 2014 & Wang J., \textbf{X. Xu}, S. Ding, and W. Hou, TEMPO \& GOES-R
synergy update and GEO-TASO aerosol retrieval, \textit{The Second TEMPO 
Science Team Meeting}, National Institute of Aerospace, Hampton, VA, 
21--22 May 2014  \\
 2013 & Spurr R., J. Wang, J. Zeng, \textbf{X. Xu}, and M. Mishchenko, Linearized
Mie and T-matrix scattering: Application in aerosol retrievals and
sensitivity studies, \textit{Electromagnetic and Light Scattering - XIV}, Lille,
France, 17--21 June 2013.\\
 2013 & \textbf{Xu X.}, J. Wang, D. K. Henze, W. Qu and M. Kopacz, �Adjoint
Inversion of Aerosol Emissions from Satellite (MODIS) Radiance
Observation with GEOS-Chem Model, \textit{the 6th International
GEOS-Chem Meeting}, Cambridge, MA, 6--9 May 2013. \\
 2012 & \textbf{Xu X.}, J. Wang, D. K. Henze, W. Qu, and M. Kopacz,
Top-Down Inversion of Aerosol Emissions through Adjoint Integration of
Satellite Radiance and GEOS-Chem Chemical Transport Model, A32B-07, 
\textit{AGU 2012 Fall Meeting}, San Francisco, CA, 3--7 December 2012. \\
 2012 & Wang J., \textbf{X. Xu}, and D. K. Henze, Toward the integrated use of
multi-sensors (MODIS, MISR, and OMI) and inverse modeling (GEOS-Chem
adjoint) to constrain the aerosol primary and precursor emissions
(invited), A24C-02, \textit{AGU 2012 Fall Meeting}, San Francisco, CA,
3--7 December 2012. \\
 2012 & Henze D. K., B. S. Meland, \textbf{X. Xu}, J. Wang, F. Akhtar, B.
Hemming, R. W. Pinder, and D. Loughlin, Remote sensing
constraints on aerosol sources, physical properties and direct radiative
forcing (Invited), A31I-02, \textit{AGU 2012 Fall Meeting}, San
Francisco, CA, 3--7 December 2012. \\
 2012 & Liu X., \textbf{X. Xu}, J. Wang, K. Chance, R. Spurr, and Y. Liu, UNL-VRTM:
A numerical testbed for remote sensing of aerosols and clouds and its
preliminary application to the TANSAT, \textit{The 1st TANSAT International
Workshop}, Beijing, China, 15--17 Oct 2012. \\
 2012 & Meland B., \textbf{X. Xu}, D. K. Henze, and J. Wang, Assessing
Top of Atmosphere Polarization Sensitivity to Aerosol Emissions Using
the GEOS-Chem Chemical Transport Model Adjoint, 6SA.5, \textit{AAAR 2012 Annual
Conference}, Minneapolis, MN, 8--12 October 2012. \\
 2012 & Wang J., \textbf{X. Xu}, D. K. Henze, and J. Zeng, Top-Down
Estimate of Dust Emissions through Integration of MODIS and MISR Aerosol
Retrievals with the GEOS-Chem Adjoint Model, 6SA.6, \textit{AAAR 2012
Annual Conference}, Minneapolis, MN, 8--12 October 2012. \\
 2012 & Meland B. S., D. K. Henze, \textbf{X. Xu}, and J. Wang, Using
the GEOS-Chem Adjoint Model to Determine the Sensitivity of Top of
Atmosphere Polarizations to Aerosol Emissions, \textit{2012 Aerosol and
Atmospheric Optics Visibility \& Air Pollution Conference}, edited,
Whitefish, MT, 24--28 September 2012. \\
 2012 & \textbf{Xu X.}, J. Wang, D. Henze, and W. Qu, Top-down inversion of aerosol
emissions over China from MODIS observed radiance with the adjoint of
GEOS-Chem chemical transport model, \textit{2012 International Conference on
Computational Science}, Omaha, NE, 4--6 June 2012. \\
 2011 & Wang J., J. Zeng, \textbf{X. Xu}, R. Spurr, X. Liu, M.
Mishchenko, B. Holben, A. Sinyuk, and Q. Han, 
AERONET Skylight Retrievals Using Polarimetric
Measurements: Toward Physically Consistent Validation of APS/RSP Aerosol
Products, \textit{2011 Glory Science Team Meeting}, NASA GISS, New York,
NY, 10--12 August 2011.\\
 2011 & Wang J., \textbf{X. Xu}, J. Zeng, X. Liu, K. Chance, and R. Spurr,
Sensitivity experiment of aerosol retrievals for GEOS-CAPE, \textit{The GEOS-CAPE
Workshop}, Boulder, CO, 11--13 May 2011. \\
 2010 & Wang J., \textbf{X. Xu}, and D. K. Henze, A new framework for the top-down
estimate of aerosol emission: Integrated analysis with satellite (MODIS)
reflectance and the adjoint of a chemistry transport model (GEOS-chem)
(Invited), \textit{AGU 2010 Fall Meeting}, San Francisco, CA, 13--17
December 2010.\\
\end{longtable}
\vskip-1em

%%% Recent Conference Presentations (Poster)
\sectionskip
{\bf\large Conference Posters}
\sectionlineskip
\hrule
\begin{longtable}{p{0.4in} p{5.8in}}
 \justifying
 2015 & \textbf{Xu X.},  J. Wang, Y. Wang, D. Henze, and L. Zhang, 
 Sensitivity of spectral climate signals to the emissions of atmospheric
dust, A31B-0028, \textit{AGU 2015 Fall Meeting}, San Francisco, CA,
14--18 December 2015. \\
 2015 & Zhu J., X. Xia, J, Wang, H. Chen, J. Zhang, \textbf{X. Xu}, M. Oo, R.
Holz, and R. Levy, Evaluation of aerosol optical depth and aerosol
models from MODIS and VIIRS retrieval algorithms over North China Plain,
A21C-0132, \textit{AGU 2015 Fall Meeting}, San Francisco, CA, 
14--18 December 2015. \\
 2015 & Hou W., J. Wang, \textbf{X. Xu}, J. Leitch, T, Delker, and G. Chen, An
algorithm for hyperspectral remote sensing of aerosols: theoretical
framework, information content analysis and application to GEO-TASO,
A11G-0123, \textit{AGU 2015 Fall Meeting}, San Francisco, CA,
14--18 December 2015. \\
 2015 & Wang Y., J. Wang, \textbf{X. Xu}, and D. Henze, Inverse estimation of
SO2 emissions over China with local air mass factor applied, A31B-0030, 
\textit{AGU 2015 Fall Meeting}, San Francisco, CA, 14--18 December 2015.\\
 2015 & \textbf{Xu X.}, J. Wang, and Y. Wang, Constraint on dust emission 
parameterization schemes through GEOS-chem adjoint and satellite data,
\textit{The 9th NASA AQAST Meeting}, St. Louis, MO, 2--4 June
2015.\\
 2015 & \textbf{Xu X.}, J. Wang, and Y. Wang, Constraining dust sources
with the MODIS and MISR observations,
\textit{The 7th International GEOS-Chem Meeting}, Cambridge, MA, 4--7
May 2015.\\
 2015 & Wang, Y., J. Wang, \textbf{X. Xu}, and D. Henze, GEOS-Chem
Adjoint Inversion of \ce{SO2} Emissions with OMI \ce{SO2} Observations over China,
\textit{The 7th International GEOS-Chem Meeting}, Cambridge, MA, 4--7
May 2015.\\
 2014 & Wang, Y., \textbf{X. Xu}, J. Wang, D. Henze, and Y. Yue, 
Observing System Simulation Experiments (OSSE) for Future Geostationary
Satellite to Constrain Aerosol Emissions through GEOS-Chem Adjoint, 
A33H-3300, \textit{AGU 2014 Fall Meeting}, San Francisco, CA, 15--19
December 2014.\\ 
 2014 & Ding, S., J. Wang, \textbf{X. Xu}, and R. Spurr, 
Retrieval of optical depth and vertical distribution of atmospheric
aerosols from light intensity and polarization in \ce{O2}-A and -B bands,
A51B-3025, \textit{AGU 2014 Fall Meeting}, San Francisco, CA, 15--19
December 2014.\\
 2014 & Han, D., J. Wang, \textbf{X. Xu}, W. Hou, and L. Chen, 
Application of GOSAT TANSO-CAI observations for aerosol optical depth
retrieval and surface PM2.5 air quality monitoring, A51B-3030,
\textit{AGU 2014 Fall Meeting}, San Francisco, CA, 15--19
December 2014.\\
 2014 & Hou, W., J. Wang, \textbf{X. Xu}, S. Ding, D. Han, J. Leitch, T. Delker,
and G. Chen, An algorithm for simultaneous inversion of aerosol
properties and surface reflectance from airborne GeoTASO hyperspectral
data, A51B-3034, \textit{AGU 2014 Fall Meeting}, San Francisco, CA, 15--19
December 2014.\\
 2014 & \textbf{Xu X.} and J. Wang, Information Content in AERONET
Photo-Polarimetric Measurements for Aerosol Microphysical Properties,
\textit{Goldschmidt2014}, Sacramento, CA, June 8--13, 2014.\\
 2014 & \textbf{Xu X.}, J. Wang, and D. Henze, First Inversion of Aerosol
Emissions from Satellite Radiances, \textit{The Goldschmidt 2014},
Sacramento, CA, June 8--13, 2014.\\
 2014 & Wang, Y., \textbf{X. Xu}, J. Wang, and D. Henze, Observing System
Simulation Experiments (OSSE) for Future Geostationary Satellite to
Constrain Aerosol Emissions, \textit{The Goldschmidt 2014},
Sacramento, CA, June 8--13, 2014. \\
 2013 & \textbf{Xu X.}, J. Wang, D. K. Henze, W. Qu, and M. Kopacz,
Inversion of Aerosol Sources from MODIS Radiances with GEOS-Chem
Adjoint, and Evaluation with Multi-Sensor Data, 
\textit{12th AEROCOM Workshop}, Hamburg, Germany, 23--27 September, 2013. \\
 2013 & \textbf{Xu X.}, J.Wang, J. Zeng, R. Spurr, X. Liu, and B. Holben, Retrieval
of aerosol microphysical properties from AERONET measurements of
polarimetric skylight radiance, \textit{Electromagnetic and Light
Scattering - XIV}, Lille, France, June 17--21, 2013.\\
 2012 & Zeng, J., J. Wang, Y. Liu, Z. Yang, and \textbf{X. Xu}, Application of
VIIRS data for remote sensing of surface particulate matter in Atlanta
city, A21C-0071, \textit{AGU 2012 Fall Meeting}, San Francisco, CA, 3--7 
December 2012. \\
 2012 & Wang, J., \textbf{X. Xu}, J. Zeng, R. J. Spurr, X. Liu, and K. Chance, 
Feasibility study for combined use of GEO-CAPE and GOES-R
observations to improve retrieval of aerosol properties, A31B-0025,
\textit{AGU 2012 Fall Meeting}, San Francisco, CA, 3--7 December 2012. \\
 2011 & \textbf{Xu X.}, J. Wang, et al., Top-down estimate of aerosol emissions
over China from MODIS reflectance by using GEOS-Chem adjoint, \textit{The 5th
International GEOS-Chem Meeting}, Cambridge, MA, 2--5 May 2011.\\
 2009 & \textbf{Xu X.} and J. Wang, Aerosol optical thickness over the east
Asia: GEOS-chem simulations constrained by MODIS reflectance, \textit{Gordon
Research Conference (GRC) on Radiation \& Climate}, New London, NH, 6--10
July 2009. \\
 2009 & \textbf{Xu X.} and J. Wang, Aerosol optical thickness over the
east Asia: GEOS-chem simulations constrained by MODIS reflectance,
\textit{The 4th International GEOS-Chem Meeting}, Cambridge, MA, 7--10 
April 2009. \\
\end{longtable}
\vskip-1em

\sectionskip
{\bf\large Services}
\sectionlineskip
\hrule
{\bf\large Reviewing for Journal Articles:} \\
(\textbf{26 reviews on 22 manuscripts for 15 journals}) \\
\textit{Aerosol and Air Quality Research}: 2015(2) \\
\textit{Atmosphere}: 2016(3) \\
\textit{Atmospheric Chemistry and Physics}: 2016(1) \\
\textit{Atmospheric Environment}: 2016(2) \\
\textit{Frontiers of Earth Science}: 2014(1) \\
\textit{International Journal of Environmental Research and Public
Health}: 2016(1) \\
\textit{International Journal of Remote Sensing}: 2014(1) \\
\textit{Journal of Atmospheric and Solar-Terrestrial Physics}:
2015(1) \\
\textit{Journal of Geophysical Research}: 2013(2) \\
\textit{Journal of Quantitative Spectroscopy and Radiative Transfer}:
2015(1), 2016(1) \\
\textit{Remote Sensing}: 2016(1) \\
\textit{Remote Sensing of Environment}: 2016(4) \\
\textit{Science of the Total Environment}: 2016(1) \\
\textit{Southeastern Geographer}: 2016(1) \\
\textit{Theoretical and Applied Climatology}: 2012(1), 2013(1), 2015(1)

{\bf\large Reviewing for Conference Papers:} \\
(\textbf{2 manuscripts for 1 conference}) \\
{\textit{The 2012 International Conference on Computational Science,
Omaha, NE 4--6 June 2012}} 

{\bf\large Memberships:}\\
Member of the \textbf{American Geophysical Union} (2010 -- present) \\
Member of the \textbf{American Meteorological Society} (2010 -- present) \\


%%%%%%%%%%%%%%%%%%%%%%%%%%%%%%
%%% Comment out
%%%%%%%%%%%%%%%%%%%%%%%%%%%%%%
\iffalse

\sectionskip
{\bf\large References}
\sectionlineskip
\hrule
\textbf{Dr. Jun Wang} \\
Associate Professor \\
Dept of Earth and Atmospheric Sciences \\
University of Nebraska-Lincoln \\
303 Bessey Hall \\
1400 R ST, Lincoln, NE 68588-0340 \\
phone: (402) 472-3597 \\
email: jwang7@unl.edu \\
(Dr. Jun Wang was my Ph.D. advisor and is my Postdoc supervisor. Jun has
known me for 7.5 years.)

\vskip1em
\textbf{Dr. Daven K. Henze} \\
Associate Professor \\
University of Colorado \\
Dept of Mechanical \\
Engineering 1111 Engineering Drive, ECME 114  \\
Boulder, Colorado, 80309-0427 \\
phone: (303) 492-8716 \\
email: daven.henze@colorado.edu

\vskip1em
\textbf{Dr. Robert J.D. Spurr} \\
Director, RT SOLUTIONS Inc. \\
9 Channing Street, \\ 
Cambridge MA 02138 \\
phone: (617) 492-1183 \\
email: rtsolutions@verizon.net

\vskip1em
\textbf{Dr. Clinton M. Rowe} \\
Professor and Graduate Chair \\
Dept of Earth and Atmospheric Sciences \\
University of Nebraska-Lincoln \\
305 Bessey Hall \\
1400 R ST, Lincoln, NE 68588-0340 \\
phone:(402)472-1946 \\
email: crowe1@unl.edu

\fi

\end{document}  
