\documentclass[11pt, letterpaper]{article}   	% use "amsart" instead of "article" for AMSLaTeX format
\usepackage[left=1.0in,top=0.9in,right=1.0in,bottom=0.9in]{geometry}                		% See geometry.pdf to learn the layout options. There are lots.
%\geometry{letterpaper}                   		% ... or a4paper or a5paper or ... 
%\geometry{landscape}                		% Activate for for rotated page geometry
%\usepackage[parfill]{parskip}    		% Activate to begin paragraphs with an empty line rather than an indent
\usepackage{graphicx}				% Use pdf, png, jpg, or eps� with pdflatex; use eps in DVI mode
								% TeX will automatically convert eps --> pdf in pdflatex		
\usepackage{amssymb}
\usepackage[usenames,dvipsnames,svgnames,table]{xcolor}
\usepackage[colorlinks = true,
            linkcolor = ProcessBlue,
            urlcolor  = ProcessBlue,
            citecolor = ProcessBlue,
            anchorcolor = ProcessBlue]{hyperref}


\usepackage[parfill]{parskip} % Remove paragraph indentation
\usepackage{array} % Required for boldface (\bf and \bfseries) tabular columns
\usepackage{ifthen} % Required for ifthenelse statements
\usepackage{url}
\pagestyle{plain}
\pagenumbering{roman}
%\pagestyle{empty} % Suppress page numbers

%this is needed to get the block text justified
\usepackage{ragged2e}
\usepackage{mathptmx}

%% For repeat
\usepackage{expl3}
\ExplSyntaxOn
\cs_new_eq:NN \Repeat \prg_replicate:nn
\ExplSyntaxOff

%% For chemical formula
\usepackage{mhchem}

%% Support to fix table width
\usepackage{array}
\newcolumntype{C}[1]{>{\centering\arraybackslash}m{#1}}
\newcolumntype{R}[1]{>{\raggedleft\arraybackslash}m{#1}}

%% For multi-page tables
\usepackage{longtable}

%\title{Xiaoguang Xu}
%\author{The Author}
%\date{}							% Activate to display a given date or no date

%%%%% 
\def\addressskip{\smallskip} % The space between the two address (or phone/email) lines
\def\sectionlineskip{\medskip} % The space above the horizontal line for each section 
\def\nameskip{\bigskip} % The space after your name at the top
\def\sectionskip{\medskip} % The space after the heading section

\begin{document}
%\maketitle
%\section{}
%\subsection{}
\vspace*{1.5em}
%\hfil{\MakeUppercase{\bf{\Large{X}\large{iaoguang (}\Large{R}\large{ichard) }\Large{ X}\large{u}}}}\hfil
\hfil{\bf{\Large{Xiaoguang (Richard) Xu}}}\hfil

\nameskip
The University of Iowa \hfill  \hfill Phone: (319) 335 $\cdot$ 1400\\
Dept. of Chem \& Biochem Engineering \hfill   Cell: (402) 805 $\cdot$ 7712\\
Center for Global \& Regional Environmental Research  \hfill
\href{xiaoguang-xu@uiowa.edu}{xiaoguang-xu@uiowa.edu}\\
IATL 401 \hfill Founder,
\href{http://www.unl-vrtm.com}{http://www.unl-vrtm.com} \\
Iowa City, IA 52242

%\vskip-1em
%\hfil\Repeat{12}{$\star\quad$}$\star$\hfil

% Enlarge the space between table rows 
%\renewcommand{\arraystretch}{1.3}
%\sectionskip
%{\bf\large Experience and Qualifications}
%\sectionlineskip
%\hrule
\vskip1em
Xiaoguang Xu's research area has focused on retrieval and data
assimilation of aerosol remote sensing data, with emphasis on the
integration of satellite remote sensing and chemistry transport model
(e.g. GEOS-Chem) to characterize aerosol distribution and emitting
sources. He developed the capability to use GEOS-Chem adjoint for
inverse modeling of dust emissions, with which he used the 4D-Var
methods to constrain speciated aerosol sources from remote sensing data.
Xiaoguang Xu is the major developer of a numerical testbed for remote
sensing of aerosol, the Unified Linearized Vector Radiative Transfer
Model (UNL-VRTM). He also developed retrieval algorithms for both 
satellite (MODIS) and ground based polarimetric (AERONET) remote
sensing platforms. He received NASA's Earth and Space Science Fellowship
in 2012.

\sectionskip
{\bf\large Education}
\sectionlineskip
\hrule
\begin{tabular}{p{0.4in} p{5.7in}}
Ph.D. & Earth and Atmospheric Sciences, University of Nebraska-Lincoln, 2015\\
      & {\small Dissertation: \textit{Retrieval of the Aerosol
                Microphysical Properties from AERONET Photopolarimetric 
                Measurements}
(\href{http://digitalcommons.unl.edu/geoscidiss/66/}{url})}\\
M.S. & Meteorology, Lanzhou University, Lanzhou, China, 2008 \\
     & {\small Thesis: \textit{Study of Predictability of the 
               T63L16 Climate Model}}  \\
B.S. &  Atmospheric Sciences, Lanzhou University, Lanzhou, China, 2005\\
\end{tabular}

%% Research Experiences
\sectionskip
{\bf\large Professional Experiences}
\sectionlineskip
\hrule
\begin{tabular}{p{1.1in} p{5.3in}}
 \justifying
09/2016--Present & Assistant Research Scientist, The University of Iowa
\\
01/2016--08/2016 & Research Assistant Professor, University of
Nebraska-Lincoln \\
09/2015--12/2015 & Postdoctoral Scholar, University of Nebraska-Lincoln \\
2010 Summer & Visiting Scholar, University of Colorado, Boulder \\ 
2009 Summer & Visiting Researcher, National Center for Atmospheric
Research (NCAR) \\
08/2008--08/2015 & Graduate Research Assistant, University of
Nebraska-Lincoln\\
07/2006--07/2008  & Graduate Researcher, National Climate Center, China
Meteorological Administration \\
\end{tabular}

%%% Teaching Experience
\sectionskip
{\bf\large Teaching Experience}
\sectionlineskip
\hrule
\begin{tabular}{p{0.9in} p{5.3in}}
 \justifying
2017 Spring & Co-Lecturer, \textit{Satellite Image Processing \& Remote
Sensing of Atmosphere}, U. Iowa\\
2015 Fall   & Guest Lecturer, \textit{Advanced Satellite Remote
Sensing}, U. Nebraska \\
2015 Fall   & Guest Lecturer, \textit{Physical Meteorology}, U. Nebraska \\
2014 Spring & Lead, \textit{Inverse Modeling \& Optimization Discussion
Series}, U. Nebraska \\
2013 Fall   & Teaching Assistant, \textit{Environment, Energy, and
Climate}, U. Nebraska \\
2012 Fall   & Guest Lecturer, \textit{Advanced Satellite Remote
Sensing}, U. Nebraska \\
2012 Spring & Teaching Assistant, \textit{Atmospheric
Thermodynamics}, U. Nebraska\\
\end{tabular}

\sectionskip
{\bf\large Awards}
\sectionlineskip
\hrule
\begin{tabular}{p{0.9in} p{5.3in}}
 \justifying 
 2014 & \textbf{Outstanding Graduate Award}, UNL Earth \& Atmospheric Sciences \\
 2012--2015 & \textbf{NASA Earth and Space Science Fellowship} \\
 2007 & \textbf{Outstanding Conference Paper}, China Meteorological
Society 2007 Annual Meeting \\
 2004 & \textbf{Outstanding Undergraduate}, Lanzhou University \\
\end{tabular} 

%% Research Grants
\sectionskip
{\bf\large Research Grants}
\sectionlineskip
\hrule

\textbf{Characterization of Global Aerosol Height Using Satellite
Observations in Oxygen A and B bands} \quad (Pending,
\$260,379, Xu is the PI). \\
\hspace{.5cm} -- NNH17ZDA001N-NIP \\
\hspace{.5cm} -- 03/15/2018 -- 03/14/2021

\textbf{Constraining Dust Refractive Index and Direct Radiative Forcing
Using Paired Sounder/Imager Sensors on Aqua and SNPP} \quad (Pending,
\$554,836, Xu is the PI). \\
\hspace{.5cm} -- NNH17ZDA001N-TASNPP \\
\hspace{.5cm} -- 12/01/2017 -- 11/30/2020

\textbf{Multi-phase inversion of aerosol sources using MODIS, MISR, OMI,
and AERONET data and the GEOS-Chem adjoint} \quad (\$809,486, Xu is
a Co-I, Dr. Daven Henze is the PI). \\
\hspace{.5cm} -- NNH16ZDA001N-ACMAP \\
\hspace{.5cm} -- 02/17/2017 -- 02/16/2020

\textbf{Sensitivity analysis and recovery of dust emissions from
spectral climate signals} \quad (\$539,721, Xu is a Co-I, Dr. Jun Wang is the PI). \\
\hspace{.5cm} -- NASA ROSES NNH14ZDA001N-ACSCS \\
\hspace{.5cm} -- 11/01/2014 -- 10/31/2017

\textbf{Constraining Global Sources of Atmospheric Mineral Dust with
Multi-Sensor Satellite Observations and the GEOS-Chem Adjoint Model}
\quad (\$90,000, Xu is the recipent of the Fellowship).\\
\hspace{.5cm} -- NASA Earth and Space Science Fellowship (NESSF12) \\
\hspace{.5cm} -- 09/01/2012 -- 08/31/2015


%%% Travel Grants
\sectionskip
{\bf\large Travel Grants}
\sectionlineskip
\hrule
\begin{longtable}{p{0.4in} p{5.8in}}
 \justifying
 2015 & The JCSDA Summer Colloquium on Satellite Data Assimilation,
Fort Collins, CO, 27 July -- 7 August 2015. \\
 2014 & The MISR Science Team Meeting, Pasadena, CA, 11--12 December 2014. \\
 2013 & The Electromagnetic and Light Scattering - XIV, Lille, France,
17--21 June 2013. \\
 2013 & The 6th International GEOS-Chem Meeting, Cambridge,
MA, 6--9 May 2013. \\
 2011 & The 5th International GEOS-Chem Meeting, Cambridge,
MA, 2--5 May 2011. \\
 2009 & The Gordon Research Conference on Radiation \& Climate,
New London, NH, 6--10 July 2009.\\
 2009 & The 4th International GEOS-Chem Meeting, Cambridge,
MA, 7--10 April 2009. \\
\end{longtable}
\vskip-1em


\sectionskip
{\bf\large Journal Publications (Total: 24)}
\sectionlineskip
\hrule
\begin{longtable}{p{0.4in} p{5.8in}}
 \justifying 
 2017 & Tao M., L. Chen, Z. Wang, J. Wang, H. Che, W. Wang, J. Tao,
\underline{X. Xu}, H. Zhu, and C. Hou, \textbf{Evaluation of MODIS Deep Blue aerosol 
algorithm in desert region of East Asia: ground validation and
inter-comparison}, \textit{J. Geophys. Res. Atmos.}, Submitted. \\
 2017 & \underline{Xu X.}, J. Wang, Y. Wang, J. Zeng, O. Torres, Y. Yang, A.
Marshak, J. Reid, and S. Miller, \textbf{Passive remote sensing of
altitude and optical depth of dust plumes using the oxygen A and B
bands: First results from EPIC/DSCOVR at Lagrange-1 point}, \textit{Geophys. 
Res. Lett.}, 44, 7544-7554, doi:\href{http://dx.doi.org/10.1002/2017GL073939}
{10.1002/2017GL073939}.\\
 2017 & \underline{Xu X.}, J. Wang, Y. Wang, D. K. Henze, L. Zhang, G. A. Grell, S.
McKeen. and B. Wielicki, \textbf{Sense size-dependent dust loading and emission
from space using reflected solar and infrared spectral measurements: An
observation system simulation experiment}, \textit{J. Geophys. Res.
Atmos.}, 122, doi:\href{http://dx.doi.org/10.1002/2017JD026677}
{10.1002/2017JD026677}.\\
 2017 & Tao, M., Z. Wang, J. Tao, L. Chen, J. Wang, C. Hou, L. Wang,
\underline{X. Xu}, and H. Zhu, \textbf{How Do Aerosol Properties Affect 
the Temporal Variation of MODIS AOD Bias in Eastern China?},
\textit{Remote Sensing}, 9(8), 800,
doi:\href{http://dx.doi.org/10.3390/rs9080800} {10.3390/rs9080800}.\\
 2017 & Wang Y., J. Wang, R. Levy, \underline{X. Xu}, and J. Reid, 
\textbf{MODIS Retrieval of Aerosol Optical Depth over Turbid Coastal
Water}, \textit{Remote Sensing}, 9(6), 595,
doi:\href{http://dx.doi.org/10.3390/rs9060595}{10.3390/rs9060595}. \\
 2017 & Zhu J., X. Xia, J. Wang, H. Che, H. Chen, 
J.  Zhang, \underline{X. Xu}, R. Levy, M. Oo, R. Holz,
M. Ayoub, \textbf{Evaluation of aerosol optical depth and aerosol models
from VIIRS retrieval algorithms over North China Plain}, \textit{Remote
Sensing}, 9(5), 432, doi:\href{http://dx.doi.org/10.3390/rs9050432}
{10.3390/rs9050432}. \\
 2017 & Qu Z., D. K. Henze, S. L. Capps, Y. Wang, \underline{X. Xu}, J.
Wang, and M. Keller, \textbf{Monthly top-down NOx emissions for China
(2005--2012): a hybrid inversion method and trend analysis}, 
\textit{J. Geophys. Res. Atmos.}, 122, 4600-4625,
doi:\href{http://dx.doi.org/10.1002/2016JD025852}{10.1002/2016JD025852}. \\
 2017 & Hou W., J. Wang, \underline{X. Xu}, and J. Reid, \textbf{An
algorithm for hyperspectral remote sensing of aerosols: 2. Information
content analysis for aerosol parameters and principal components of
surface spectra}, \textit{Journal of Quantitative
Spectroscopy and Radiative Transfer}, 192, 14--29,
doi:\href{http://dx.doi.org/10.1016/j.jqsrt.2017.01.041}
{http://dx.doi.org/10.1016/j.jqsrt.2017.01.041}. \\
 2017 & Chen X., J. Wang, Y. Liu, \underline{X. Xu}, Z. Cai, D. Yang,
and C. Yan, \textbf{Angular dependence of aerosol information content in
CAPI/TanSat observation over land: effect of polarization and synergy
with A-train satellites}, \textit{Remote Sensing of
Environment}, 196, 163-177, 
doi:\href{https://doi.org/10.1016/j.rse.2017.05.007}
{https://doi.org/10.1016/j.rse.2017.05.007}. \\
 2016 & Wang Y., J. Wang, \underline{X. Xu}, D. K. Henze, Y. Wang, and Z. 
Qu, \textbf{A new approach for monthly updates of anthropogenic sulfur dioxide
emissions from space: implications for air quality forecasts}, 
\textit{Geophys. Res. Lett.}, 2016, 9931--9938,
doi:\href{http://dx.doi.org/10.1002/2016GL070204}
{10.1002/2016GL070204}. \\
 2016 & Wang J., C. Aegerter, \underline{X. Xu}, and J. Szykman, \textbf{Potential
application of VIIRS Day/Night Band for monitoring nighttime surface
PM2.5 air quality from space}, \textit{Atmospheric Environment}, 2016,
124, 55--63, doi:\href{https://doi.org/10.1016/j.atmosenv.2015.11.013}
{https://doi.org/10.1016/j.atmosenv.2015.11.013}.\\
 2016 & Hou W., J. Wang, \underline{X. Xu}, and J. Reid, \textbf{An algorithm 
for hyperspectral remote sensing of aerosols: 1. Development of 
theoretical framework}, \textit{Journal of Quantitative
Spectroscopy and Radiative Transfer}, 2016, 178, 400--415, 
doi:\href{https://doi.org/10.1016/j.jqsrt.2016.01.019}
{https://doi.org/10.1016/j.jqsrt.2016.01.019}. \\
 2016 & Ding S., J. Wang, and \underline{X. Xu}, \textbf{Polarimetric remote
sensing in O2 A and B bands: Sensitivity study and information content
analysis for vertical profile of aerosols}, \textit{Atmos. Meas. Tech.}, 
2016, 9, 2077--2092, doi:\href{https://doi.org/10.5194/amt-9-2077-2016}
{10.5194/amt-9-2077-2016}. \\
 2015 & \underline{Xu X.} and and J. Wang, \textbf{Retrieval of aerosol
microphysical properties from AERONET photopolarimetric measurements:
1. Information content analysis}, \textit{J. Geophys. Res. Atmos.}, 2015,
120, 7059--7078, doi:\href{http://dx.doi.org/10.1002/2015JD023108}
{10.1002/2015JD023108} \\
 2015 & \underline{Xu X.},  J. Wang, J. Zeng, R. Spurr, X. Liu, O. Dubovik,
L. Li, Z. Li, M. I. Mishchenko, A. Siniuk, and B. N. Holben, \textbf{
Retrieval of aerosol microphysical properties from AERONET photopolarimetric
measurements: 2. A new research algorithm and case demonstration},
\textit{J. Geophys. Res. Atmos.}, 2015, 120, 7079--7098,
doi:\href{http://dx.doi.org/10.1002/2015JD023113} {10.1002/2015JD023113}.\\ 
2014 & Wang J., \underline{X. Xu}, S. Ding, J. Zeng, R. Spurr, X. Liu, K. Chance,
and M. Mishchenko, \textbf{A numerical testbed for remote sensing of aerosols,
and its demonstration for evaluating retrieval synergy from a geostationary 
satellite constellation of GEO-CAPE and GOES-R}, \textit{Journal of 
Quantitative Spectroscopy and Radiative Transfer}, 2014, 146(0),
510--528, doi:\href{https://doi.org/10.1016/j.jqsrt.2014.03.020}
{https://doi.org/10.1016/j.jqsrt.2014.03.020}. \\
2013 & Meland B. S., \underline{X. Xu}, D. Henze, and J. Wang, \textbf{Assessing
remote polarimetric measurement sensitivities to aerosol emissions using
the GEOS-Chem adjoint model}, \textit{Atmos. Meas. Tech.}, 2013, 6,
3441--3457, doi:\href{https://doi.org/10.5194/amt-6-3441-2013}
{10.5194/amt-6-3441-2013}. \\ 
2013 & \underline{Xu X.}, J. Wang, D. K. Henze, W. Qu, and M. Kopacz,
\textbf{Constraints on aerosol sources using GEOS-Chem adjoint and MODIS
radiances, and evaluation with multisensor (OMI, MISR) data}, \textit{J.
Geophys. Res. Atmos.}, 2013, 118(12), 6396--6413,
doi:\href{http://dx.doi.org/10.1002/jgrd.50515}
{10.1002/jgrd.50515}. \\
2013 & Wang C., J. Li, and \underline{X. Xu}, \textbf{Wavelet analysis of
quasi-3-year temperature oscillations in China in last 50 years, and
predicted changes in the next 20 years}, \textit{Sciences in Cold and
Arid Regions}, 2013, 5(6), 0755--0766. \\ 
2012 & Wang J., \underline{X. Xu}, D. K. Henze, J. Zeng, Q. Ji, S.-C.
Tsay, and J. Huang, \textbf{Top-down estimate of dust emissions through
integration of MODIS and MISR aerosol retrievals with the GEOS-Chem
adjoint model}, \textit{Geophys. Res. Lett.}, 2012, 39(8), L08802,
doi:\href{http://dx.doi.org/10.1029/2012GL051136} {10.1029/2012GL051136}. \\
2012 & Wang C., J. Li, X. Li, and \underline{X. Xu}, Analysis on
Quasi-periodic Characteristics of Precipitation in Recent 50 Years and
Trend in Next 20 Years in China. \textit{Arid Zone Research}, 2012, 1, 002. \\
2012 & Wang C., J. Li, and \underline{X. Xu}, Universality of Quasi-3-year
Period of Temperature in Last 50 Years and Change in Next 20 Years in
China, \textit{Plateau Meteorology}, 2012, 31(1), 126--136. \\
2010 & Wang J., \underline{X. Xu}, R. Spurr, Y. Wang, and E. Drury,
\textbf{Improved algorithm for MODIS satellite retrievals of aerosol optical
thickness over land in dusty atmosphere: Implications for air quality
monitoring in China}, \textit{Remote Sensing of Environment}, 2010, 114(11),
2575--2583, doi:\href{https://doi.org/10.1016/j.rse.2010.05.034}
{https://doi.org/10.1016/j.rse.2010.05.034}. \\
2009 & \underline{Xu X.}, W. Li, H. Ren, and P. Zhang, Distribution of
prediction capacity of T63L16 model for medium-range forecast at
different spatial scales, \textit{Acta Meteorologica Sinica}, 2009, 67(6),
992--1001.\\
\end{longtable} 
\vskip-1em
%\quad Total citation: \textbf{182} (Google Scholar Citations as of June 2016) 

\sectionskip
{\bf\large Book Chapters}
\sectionlineskip
\hrule
\begin{longtable}{p{0.4in} p{5.8in}}
 \justifying
2017 & \underline{Xu X.}, J. Wang, Y. Wang, and A. Kokhanovsky,
\textbf{Passive remote sensing of aerosol height}, in
\textit{Remote Sensing of Aerosols, Clouds, and Precipitation}, edited
by T. Islam, Y. Hu, A. Kokhanovsky, and J. Wang, pp. 1-22, Elsevier,
Cambridge, MA. In press. \\

\end{longtable}
\vskip-1em


%%% Recent Conference Presentations (Oral) 
\sectionskip
{\bf\large Conference Talks}
\sectionlineskip
\hrule
\begin{longtable}{p{0.4in} p{5.8in}}
 \justifying
 2017 & \underline{Xu X.} and J. Wang, Multi-sensor aerosol retrievals,
\textit{MURI Annual Meeting}, Fort Collins, CO, 8-9 August 2017. \\
 2017 & \underline{Xu X.} and J. Wang, A pilot study of spectral
fingerprints of absorbing aerosols above boundary layer clouds,
\textit{CLARREO Science Definition Team Meeting}, Boulder, CO, May
17-19th, 2017. \\
 2017 & Wang J., S. Ding, and \underline{X. Xu}, Polarimetric remote sensing in
oxygen A and B bands: sensitivity study and information content analysis
for vertical profile of aerosols,  \textit{The 16th Electromagnetic and
Light Scattering Conference}, College Park, MD, 19-25 March 2017. \\
 2017 & Hou W, J. Wang, and \underline{X. Xu}, Feasibility analysis of hyperspectral
remote sensing of aerosols from future geostationary satellites,
\textit{The 16th Electromagnetic and
Light Scattering Conference}, College Park, MD, 19-25 March 2017. \\
 2016 & Chen X., J. Wang, \underline{X. Xu}, Y. Liu, D. Yang, and Z. Cai, Angular
dependence and the role of polarization for aerosol microphysical
properties information content from simulated CAPI/TanSat observation
over land, A32A-05, \textit{AGU 2016 Fall Meeting}, San Francisco, CA,
12--16 December 2016. \\
 2016 & \underline{Xu X.}, J. Wang, Y. Wang, D. Henze, and L. Zhang, Assessing
Information Content of CLARREO Measurements to Size-Dependent Dust
Emissions: An OSSE Study, \textit{CLARREO Science Definition Team
Meeting}, National Institute of Aerospace, Hampton, VA, Nov 29 -- Dec 1,
2016\\ 
 2016 & Wang J., \underline{X. Xu}, D. Henze, and L. Zhang, Assessing Information
Content of CLARREO Measurements to Size-Dependent Dust Emissions: An
OSSE Study, \textit{CLARREO Science Definition Team Meeting}, Ann Arbor,
MI, 10--12 May 2016. \\
 2015 & Wang J., S. Ding, and \underline{X. Xu}, Polarimetric remote
sensing in O2 A and B bands: Sensitivity study and information content
analysis for vertical profile of aerosols, A21L-06, \textit{AGU 2015 Fall
Meeting}, San Francisco, CA, 14--18 December 2015. \\
 2015 & Wang J., Y. Yue, \underline{X. Xu}, Y. Liu, R. Levy, J. J. Szykman,
and R. Holz, Will ensemble approach improve surface PM2.5 estimate from
space? \textit{MODIS/VIIRS Science Team Meeting}, Silver Spring, MD,
18--22 May 2015. \\
 2015 & Wang J., \underline{X. Xu}, Y. Wang, D. Henze, and L. Zhang, 
 Dust Emission Optimization with Satellite Remote Sensing: Application
to CLAEERO, \textit{CLARREO Science Definition Team Meeting}, Lawrence
Berkeley National Laboratory, CA, 28--30 April 2015. \\
 2014 & \underline{Xu X.}, J. Wang, J. Zeng, R. Spurr, X. Liu, O. Dubovik, Z. Li,
L. Li, B. Holben, and M. Mishchenko, An algorithm for retrieving fine
and coarse aerosol microphysical properties from AERONET-type
photopolarimetric measurements, A54B-04, \textit{AGU 2014 Fall Meeting}, San
Francisco, CA, 15--19 December 2014. \\
 2014 & \underline{Xu X.}, J. Wang, Y. Wang, and D. Henze, Adjoint Inversion of 
Atmospheric Dust Sources with the MODIS and MISR Observations, 
\textit{2014 MISR Science Team Meeting}, Pasadena, CA, 11--12 December 2014. \\
 2014 & Henze D., L. Zhang, L. Zhu, \underline{X. Xu}, J. Wang, K. Cady-Pereira,
M. Shephard, J. Bash, C. Lee, and R. Martin, Keynote: Remote Sensing
Constraints on Aerosol Sources and Impacts, \textit{Goldschmidt2014}, Sacramento,
CA, 8--13 June 2014. \\
 2014 & Wang J., \underline{X. Xu}, S. Ding, and W. Hou, TEMPO \& GOES-R
synergy update and GEO-TASO aerosol retrieval, \textit{The Second TEMPO 
Science Team Meeting}, National Institute of Aerospace, Hampton, VA, 
21--22 May 2014  \\
 2013 & Spurr R., J. Wang, J. Zeng, \underline{X. Xu}, and M. Mishchenko, Linearized
Mie and T-matrix scattering: Application in aerosol retrievals and
sensitivity studies, \textit{Electromagnetic and Light Scattering - XIV}, Lille,
France, 17--21 June 2013.\\
 2013 & \underline{Xu X.}, J. Wang, D. K. Henze, W. Qu and M. Kopacz, �Adjoint
Inversion of Aerosol Emissions from Satellite (MODIS) Radiance
Observation with GEOS-Chem Model, \textit{the 6th International
GEOS-Chem Meeting}, Cambridge, MA, 6--9 May 2013. \\
 2012 & \underline{Xu X.}, J. Wang, D. K. Henze, W. Qu, and M. Kopacz,
Top-Down Inversion of Aerosol Emissions through Adjoint Integration of
Satellite Radiance and GEOS-Chem Chemical Transport Model, A32B-07, 
\textit{AGU 2012 Fall Meeting}, San Francisco, CA, 3--7 December 2012. \\
 2012 & Wang J., \underline{X. Xu}, and D. K. Henze, Toward the integrated use of
multi-sensors (MODIS, MISR, and OMI) and inverse modeling (GEOS-Chem
adjoint) to constrain the aerosol primary and precursor emissions
(invited), A24C-02, \textit{AGU 2012 Fall Meeting}, San Francisco, CA,
3--7 December 2012. \\
 2012 & Henze D. K., B. S. Meland, \underline{X. Xu}, J. Wang, F. Akhtar, B.
Hemming, R. W. Pinder, and D. Loughlin, Remote sensing
constraints on aerosol sources, physical properties and direct radiative
forcing (Invited), A31I-02, \textit{AGU 2012 Fall Meeting}, San
Francisco, CA, 3--7 December 2012. \\
 2012 & Liu X., \underline{X. Xu}, J. Wang, K. Chance, R. Spurr, and Y. Liu, UNL-VRTM:
A numerical testbed for remote sensing of aerosols and clouds and its
preliminary application to the TANSAT, \textit{The 1st TANSAT International
Workshop}, Beijing, China, 15--17 Oct 2012. \\
 2012 & Meland B., \underline{X. Xu}, D. K. Henze, and J. Wang, Assessing
Top of Atmosphere Polarization Sensitivity to Aerosol Emissions Using
the GEOS-Chem Chemical Transport Model Adjoint, 6SA.5, \textit{AAAR 2012 Annual
Conference}, Minneapolis, MN, 8--12 October 2012. \\
 2012 & Wang J., \underline{X. Xu}, D. K. Henze, and J. Zeng, Top-Down
Estimate of Dust Emissions through Integration of MODIS and MISR Aerosol
Retrievals with the GEOS-Chem Adjoint Model, 6SA.6, \textit{AAAR 2012
Annual Conference}, Minneapolis, MN, 8--12 October 2012. \\
 2012 & Meland B. S., D. K. Henze, \underline{X. Xu}, and J. Wang, Using
the GEOS-Chem Adjoint Model to Determine the Sensitivity of Top of
Atmosphere Polarizations to Aerosol Emissions, \textit{2012 Aerosol and
Atmospheric Optics Visibility \& Air Pollution Conference}, edited,
Whitefish, MT, 24--28 September 2012. \\
 2012 & \underline{Xu X.}, J. Wang, D. Henze, and W. Qu, Top-down inversion of aerosol
emissions over China from MODIS observed radiance with the adjoint of
GEOS-Chem chemical transport model, \textit{2012 International Conference on
Computational Science}, Omaha, NE, 4--6 June 2012. \\
 2011 & Wang J., J. Zeng, \underline{X. Xu}, R. Spurr, X. Liu, M.
Mishchenko, B. Holben, A. Sinyuk, and Q. Han, 
AERONET Skylight Retrievals Using Polarimetric
Measurements: Toward Physically Consistent Validation of APS/RSP Aerosol
Products, \textit{2011 Glory Science Team Meeting}, NASA GISS, New York,
NY, 10--12 August 2011.\\
 2011 & Wang J., \underline{X. Xu}, J. Zeng, X. Liu, K. Chance, and R. Spurr,
Sensitivity experiment of aerosol retrievals for GEOS-CAPE, \textit{The GEOS-CAPE
Workshop}, Boulder, CO, 11--13 May 2011. \\
 2010 & Wang J., \underline{X. Xu}, and D. K. Henze, A new framework for the top-down
estimate of aerosol emission: Integrated analysis with satellite (MODIS)
reflectance and the adjoint of a chemistry transport model (GEOS-chem)
(Invited), \textit{AGU 2010 Fall Meeting}, San Francisco, CA, 13--17
December 2010.\\
\end{longtable}
\vskip-1em

%%% Recent Conference Presentations (Poster)
\sectionskip
{\bf\large Conference Posters}
\sectionlineskip
\hrule
\begin{longtable}{p{0.4in} p{5.8in}}
 \justifying
 2017 & Xu X. and J. Wang, Constraints from reflected solar and infrared
spectral measurements on size-dependent dust emissions: An OSSE using
FIM-Chem and GEOS-Chem, \textit{The 8th International GEOS-Chem
Meeting}, Cambridge, MA, 1-4 May 2017. \\
 2016 & Qu Z., D. Henze, S. Capps, Y. Wang, \underline{X. Xu}, J. Wang, and
M. Keller, Decadal-scale joint inversion of NOx and SO2 using a hybrid
4D-Var \& mass balance approach, A31E-0084, \textit{AGU 2016 Fall
Meeting}, San Francisco, CA, 12--16 December 2016. \\
 2016 & Wang Y., J. Wang, \underline{X. Xu}, and R. Levy, Retrieving
Aerosol Optical Depth over Turbid Coastal Water, A41A-0009, \textit{AGU
2016 Fall Meeting}, San Francisco, CA, 12--16 December 2016. \\
 2016 & \underline{Xu X.},  J. Wang, Y. Wang, and Y. Yue, Retrieval of dust
plume altitude using the DSCOVR/EPIC measurements in the oxygen-A and -B
bands, A23D-0264, \textit{AGU 2016 Fall Meeting}, San Francisco, CA,
12--16 December 2016. \\
 2015 & \underline{Xu X.},  J. Wang, Y. Wang, D. Henze, and L. Zhang, 
 Sensitivity of spectral climate signals to the emissions of atmospheric
dust, A31B-0028, \textit{AGU 2015 Fall Meeting}, San Francisco, CA,
14--18 December 2015. \\
 2015 & Zhu J., X. Xia, J, Wang, H. Chen, J. Zhang, \underline{X. Xu}, M. Oo, R.
Holz, and R. Levy, Evaluation of aerosol optical depth and aerosol
models from MODIS and VIIRS retrieval algorithms over North China Plain,
A21C-0132, \textit{AGU 2015 Fall Meeting}, San Francisco, CA, 
14--18 December 2015. \\
 2015 & Hou W., J. Wang, \underline{X. Xu}, J. Leitch, T, Delker, and G. Chen, An
algorithm for hyperspectral remote sensing of aerosols: theoretical
framework, information content analysis and application to GEO-TASO,
A11G-0123, \textit{AGU 2015 Fall Meeting}, San Francisco, CA,
14--18 December 2015. \\
 2015 & Wang Y., J. Wang, \underline{X. Xu}, and D. Henze, Inverse estimation of
SO2 emissions over China with local air mass factor applied, A31B-0030, 
\textit{AGU 2015 Fall Meeting}, San Francisco, CA, 14--18 December 2015.\\
 2015 & \underline{Xu X.}, J. Wang, and Y. Wang, Constraint on dust emission 
parameterization schemes through GEOS-chem adjoint and satellite data,
\textit{The 9th NASA AQAST Meeting}, St. Louis, MO, 2--4 June
2015.\\
 2015 & \underline{Xu X.}, J. Wang, and Y. Wang, Constraining dust sources
with the MODIS and MISR observations,
\textit{The 7th International GEOS-Chem Meeting}, Cambridge, MA, 4--7
May 2015.\\
 2015 & Wang, Y., J. Wang, \underline{X. Xu}, and D. Henze, GEOS-Chem
Adjoint Inversion of \ce{SO2} Emissions with OMI \ce{SO2} Observations over China,
\textit{The 7th International GEOS-Chem Meeting}, Cambridge, MA, 4--7
May 2015.\\
 2014 & Wang, Y., \underline{X. Xu}, J. Wang, D. Henze, and Y. Yue, 
Observing System Simulation Experiments (OSSE) for Future Geostationary
Satellite to Constrain Aerosol Emissions through GEOS-Chem Adjoint, 
A33H-3300, \textit{AGU 2014 Fall Meeting}, San Francisco, CA, 15--19
December 2014.\\ 
 2014 & Ding, S., J. Wang, \underline{X. Xu}, and R. Spurr, 
Retrieval of optical depth and vertical distribution of atmospheric
aerosols from light intensity and polarization in \ce{O2}-A and -B bands,
A51B-3025, \textit{AGU 2014 Fall Meeting}, San Francisco, CA, 15--19
December 2014.\\
 2014 & Han, D., J. Wang, \underline{X. Xu}, W. Hou, and L. Chen, 
Application of GOSAT TANSO-CAI observations for aerosol optical depth
retrieval and surface PM2.5 air quality monitoring, A51B-3030,
\textit{AGU 2014 Fall Meeting}, San Francisco, CA, 15--19
December 2014.\\
 2014 & Hou, W., J. Wang, \underline{X. Xu}, S. Ding, D. Han, J. Leitch, T. Delker,
and G. Chen, An algorithm for simultaneous inversion of aerosol
properties and surface reflectance from airborne GeoTASO hyperspectral
data, A51B-3034, \textit{AGU 2014 Fall Meeting}, San Francisco, CA, 15--19
December 2014.\\
 2014 & \underline{Xu X.} and J. Wang, Information Content in AERONET
Photo-Polarimetric Measurements for Aerosol Microphysical Properties,
\textit{Goldschmidt2014}, Sacramento, CA, June 8--13, 2014.\\
 2014 & \underline{Xu X.}, J. Wang, and D. Henze, First Inversion of Aerosol
Emissions from Satellite Radiances, \textit{The Goldschmidt 2014},
Sacramento, CA, June 8--13, 2014.\\
 2014 & Wang, Y., \underline{X. Xu}, J. Wang, and D. Henze, Observing System
Simulation Experiments (OSSE) for Future Geostationary Satellite to
Constrain Aerosol Emissions, \textit{The Goldschmidt 2014},
Sacramento, CA, June 8--13, 2014. \\
 2013 & \underline{Xu X.}, J. Wang, D. K. Henze, W. Qu, and M. Kopacz,
Inversion of Aerosol Sources from MODIS Radiances with GEOS-Chem
Adjoint, and Evaluation with Multi-Sensor Data, 
\textit{12th AEROCOM Workshop}, Hamburg, Germany, 23--27 September, 2013. \\
 2013 & \underline{Xu X.}, J.Wang, J. Zeng, R. Spurr, X. Liu, and B. Holben, Retrieval
of aerosol microphysical properties from AERONET measurements of
polarimetric skylight radiance, \textit{Electromagnetic and Light
Scattering - XIV}, Lille, France, June 17--21, 2013.\\
 2012 & Zeng, J., J. Wang, Y. Liu, Z. Yang, and \underline{X. Xu}, Application of
VIIRS data for remote sensing of surface particulate matter in Atlanta
city, A21C-0071, \textit{AGU 2012 Fall Meeting}, San Francisco, CA, 3--7 
December 2012. \\
 2012 & Wang, J., \underline{X. Xu}, J. Zeng, R. J. Spurr, X. Liu, and K. Chance, 
Feasibility study for combined use of GEO-CAPE and GOES-R
observations to improve retrieval of aerosol properties, A31B-0025,
\textit{AGU 2012 Fall Meeting}, San Francisco, CA, 3--7 December 2012. \\
 2011 & \underline{Xu X.}, J. Wang, et al., Top-down estimate of aerosol emissions
over China from MODIS reflectance by using GEOS-Chem adjoint, \textit{The 5th
International GEOS-Chem Meeting}, Cambridge, MA, 2--5 May 2011.\\
 2009 & \underline{Xu X.} and J. Wang, Aerosol optical thickness over the east
Asia: GEOS-chem simulations constrained by MODIS reflectance, \textit{Gordon
Research Conference (GRC) on Radiation \& Climate}, New London, NH, 6--10
July 2009. \\
 2009 & \underline{Xu X.} and J. Wang, Aerosol optical thickness over the
east Asia: GEOS-chem simulations constrained by MODIS reflectance,
\textit{The 4th International GEOS-Chem Meeting}, Cambridge, MA, 7--10 
April 2009. \\
\end{longtable}
\vskip-1em

\sectionskip
{\bf\large Services}
\sectionlineskip
\hrule
{\bf\large Reviewing for Journal Articles:} \\
%(\textbf{30 reviews on 26 manuscripts for 15 journals}) \\
\textit{Aerosol and Air Quality Research}: 2015(2), 2016(1) \\
\textit{Atmosphere}: 2016(4) \\
\textit{Atmospheric Chemistry and Physics}: 2016(1), 2017(1) \\
\textit{Atmospheric Environment}: 2016(2) \\
\textit{Frontiers of Earth Science}: 2014(1) \\
\textit{Geoscientific Model Development}: 2017(1) \\
\textit{International Journal of Environmental Research and Public
Health}: 2016(1) \\
\textit{International Journal of Remote Sensing}: 2014(1) \\
\textit{Journal of Atmospheric and Solar-Terrestrial Physics}:
2015(1), 2016(1) \\
\textit{Journal of Geophysical Research}: 2013(2), 2016(1), 2017(4) \\
\textit{Journal of Quantitative Spectroscopy and Radiative Transfer}:
2015(1), 2016(1) \\
\textit{Remote Sensing}: 2016(1), 2017(6) \\
\textit{Remote Sensing of Environment}: 2016(5), 2017(3) \\
\textit{Science of the Total Environment}: 2016(2), 2017(1) \\
\textit{Southeastern Geographer}: 2016(1) \\
\textit{Theoretical and Applied Climatology}: 2012(1), 2013(1), 2015(1)
\\
\textit{Water, Air, \& Soil Pollution}: 2017(1)

{\bf\large Reviewing for Conference Papers:} \\
%(\textbf{2 manuscripts for 1 conference}) \\
{\textit{The 2012 International Conference on Computational Science,
Omaha, NE 4--6 June 2012}} 

{\bf\large Service to the University:} \\
{\textit{Reviewing for applications to U. Nebraska-Lincoln UCARE
program, 2016}

{\bf\large Professional Memberships:}\\
Member of the \textbf{American Geophysical Union} (2010 -- present) \\
Member of the \textbf{American Meteorological Society} (2010 -- present) \\



%%%%%%%%%%%%%%%%%%%%%%%%%%%%%%
%%% Comment out
%%%%%%%%%%%%%%%%%%%%%%%%%%%%%%
\iffalse

%%%% Reference section
\sectionskip
{\bf\large References}
\sectionlineskip
\hrule
\textbf{Dr. Jun Wang} \\
Associate Professor \\
Dept of Earth and Atmospheric Sciences \\
University of Nebraska-Lincoln \\
303 Bessey Hall \\
1400 R ST, Lincoln, NE 68588-0340 \\
phone: (402) 472-3597 \\
email: jwang7@unl.edu \\
(Dr. Jun Wang was my Ph.D. advisor and is my Postdoc supervisor. Jun has
known me for 7.5 years.)

\vskip1em
\textbf{Dr. Daven K. Henze} \\
Associate Professor \\
University of Colorado \\
Dept of Mechanical \\
Engineering 1111 Engineering Drive, ECME 114  \\
Boulder, Colorado, 80309-0427 \\
phone: (303) 492-8716 \\
email: daven.henze@colorado.edu

\vskip1em
\textbf{Dr. Robert J.D. Spurr} \\
Director, RT SOLUTIONS Inc. \\
9 Channing Street, \\ 
Cambridge MA 02138 \\
phone: (617) 492-1183 \\
email: rtsolutions@verizon.net

\vskip1em
\textbf{Dr. Clinton M. Rowe} \\
Professor and Graduate Chair \\
Dept of Earth and Atmospheric Sciences \\
University of Nebraska-Lincoln \\
305 Bessey Hall \\
1400 R ST, Lincoln, NE 68588-0340 \\
phone:(402)472-1946 \\
email: crowe1@unl.edu

\fi

\end{document}  
